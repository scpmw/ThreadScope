% $Id: ghc-parallel-tuning.tex#1 2009/04/22 10:31:00 REDMOND\\satnams $
% $Source: //depot/satnams/haskell/ThreadScope/papers/haskell_symposium_2009/ghc-parallel-tuning.tex $

\documentclass[twocolumn,9pt]{sigplanconf}

\usepackage{url}
% \usepackage{code}
\usepackage{graphicx}
\usepackage{enumerate}


\nocaptionrule

\title{Parallel Performance Tuning for Haskell}

\authorinfo{Donnie Jones}{TBD}
           {donnie@darthik.com}
\authorinfo{Simon Marlow}{Microsoft Research}
           {simonmar@microsoft.com}
\authorinfo{Satnam Singh}{Microsoft Research}
           {satnams@microsoft.com}

\begin{document}

\maketitle
%\makeatactive

\begin{abstract}
Profiling information is essential for the performance tuning of parallel programs. This paper describes a new parallel profiling mechansim in the GHC system and its associated graphical viewer. We illustrate how this system can be used to help diagnose and fix several common kinds of performance errors in semi-explicit parallel Haskell programs.
\end{abstract}


\section{Introduction}

\section{Background}

\section{Profiling Motivation}
Show examples of semi-explicit parallel programs that go wrong. Show what we could measure before using heap and time profiling and motivate the need for better profiling.

\section{Profiling Infrastructure}
The basic infrastrucutre of the systsem, design choices and justification and a little about the viewer ThreadScope.

\section{Case Studies}
Revist examples and show how performance can be tuned with the information obtained from the profiler.

\subsection{SumEuler}
This example has been used several times before and it is useful for illustrating how sparks should be carefully constructed and the need for pseq.

% $Id: related-work.tex#1 2009/04/22 10:31:00 REDMOND\\satnams $
% $Source: //depot/satnams/haskell/ThreadScope/papers/haskell_symposium_2009/related-work.tex $

\section{Related Work}
GranSim~\cite{loidl} is an event-driven simulator for the parallel execution of Glasgow Parallel Haskell (GPH) programs which allows the parallel behaviour of Haskell programs to be analyzed by instantiating any number of virtual processors which are emulated by a single thread on the host machine. GranSim has an associated set of visualization tools which show overall activity, per-processor activity, and per-thread activity. There is also a sepatate tool for analyzing the granularity of the generated threads.

\section{Conclusions}

\bibliographystyle{alpha}
\bibliography{ghc-parallel-tuning}

\end{document}
